%
% LaTeX template for prepartion of submissions to PLDI'15
%
% Requires temporary version of sigplanconf style file provided on
% PLDI'15 web site.
% 
\documentclass[pldi]{sigplanconf-pldi15}

%
% the following standard packages may be helpful, but are not required
%
\usepackage{SIunits}            % typset units correctly
\usepackage{courier}            % standard fixed width font
\usepackage[scaled]{helvet} % see www.ctan.org/get/macros/latex/required/psnfss/psnfss2e.pdf
\usepackage{url}                  % format URLs
\usepackage{listings}          % format code
\usepackage{enumitem}      % adjust spacing in enums
\usepackage[colorlinks=true,allcolors=blue,breaklinks,draft=false]{hyperref}   % hyperlinks, including DOIs and URLs in bibliography
% known bug: http://tex.stackexchange.com/questions/1522/pdfendlink-ended-up-in-different-nesting-level-than-pdfstartlink
\newcommand{\doi}[1]{doi:~\href{http://dx.doi.org/#1}{\Hurl{#1}}}   % print a hyperlinked DOI



\begin{document}

%
% any author declaration will be ignored  when using 'plid' option (for double blind review)
%

\title{Improb : Probabilistic Improvisation}

\maketitle
\begin{abstract}
improb is a simple music generation langauge
intended to generate riffs that change

\end{abstract}

\section{Introduction}
An Improb program represents a walk through a variable-memory markov chain.

\paragraph{Users}

\paragraph{Background}

\section{Goals}

\section{Improb 101}

\subsection{Features}

\subsection{Syntax}

\subsection{Example}

\subsection{Output}

\section{How it works}

\subsection{Simplification}

\subsection{Markov Chain}

\subsection{Euterpea}

\section{Wish List}
\paragraph{Tools}
\paragraph{Future}

\section{Evaluation}

\section{Acknowledgements}



\bibliographystyle{abbrvnat}

% We recommend that you use BibTeX.  The inlined bibitems below are
% used to keep this template to a single file.
\begin{thebibliography}{}
\softraggedright

\bibitem[Backus et~al.(1960)]{Backus:60}
J.~W. Backus, F.~L. Bauer, J.~Green, C.~Katz, J.~McCarthy, A.~J. Perlis,
  H.~Rutishauser, K.~Samelson, B.~Vauquois, J.~H. Wegstein, A.~van Wijngaarden,
  and M.~Woodger.
\newblock Report on the algorithmic language ALGOL 60.
\newblock \emph{Commun. ACM}, 3\penalty0 (5):\penalty0 299--314, May 1960.
\newblock ISSN 0001-0782.
\newblock \doi{10.1145/367236.367262}.

\bibitem[Collins(1960)]{Collins:60}
G.~E. Collins.
\newblock A method for overlapping and erasure of lists.
\newblock \emph{Commun. ACM}, 3\penalty0 (12):\penalty0 655--657, December
  1960.
\newblock \doi{10.1145/367487.367501}

\bibitem[Lamport(1994)]{Lamport:94}
L.~Lamport.
\newblock \emph{{\LaTeX: A Document Preparation System}}.
\newblock Addison-Wesley, Reading, Massachusetts, 2nd edition, 1994.

\bibitem[McCarthy(1960)]{McCarthy:60}
J.~McCarthy.
\newblock Recursive functions of symbolic expressions and their computation by
  machine, part {I}.
\newblock \emph{Commun. ACM}, 3\penalty0 (4):\penalty0 184--195, April 1960.
\newblock \doi{10.1145/367177.367199}

\end{thebibliography}

\end{document}
